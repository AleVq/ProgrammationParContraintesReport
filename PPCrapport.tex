\documentclass[french]{article}
\usepackage[T1]{fontenc}
\usepackage[utf8]{inputenc}
\usepackage{lmodern}
\usepackage[a4paper]{geometry}
\usepackage{babel}
\usepackage{enumitem}   
\usepackage{amssymb}
\usepackage{amsthm}
\usepackage{amsmath}
\usepackage{algorithm}
\usepackage{hyperref}
\usepackage{algpseudocode}
\usepackage{subcaption}
\usepackage{graphicx}
\theoremstyle{definition}
\newtheorem{deff}{D\'efinition}[section]
\newtheorem{rem}{Remarque}[section]
\graphicspath{{./images/}}
\theoremstyle{remark}
\newtheorem*{remark}{Remark}


\begin{document}
	\title{Programmation par Contraintes - Rapport}
	\author{Vasquez Alessandro}
	\maketitle
	\newpage
	\tableofcontents
	\newpage
\section{Introduction}
\label{sec:intro}
\begin{figure*}[t!]
	\begin{subfigure}[t]{0.5\textwidth}
		\centering
		\includegraphics[height=1.2in]{isomorphismesimple.eps}
		\caption{\label{fig:isographstar}Deux graphes isomorphes}
	\end{subfigure}%
	~ 
	\begin{subfigure}[t]{0.5\textwidth}
		\centering
		\includegraphics[height=1.2in]{nonisomorphegraphes.eps}
		\caption{\label{fig:nonisograph}Graphes  non-isomorphes}
	\end{subfigure}
	\caption{Isomorphisme, examples}
\end{figure*}

Nous verrons d'abord les définitions des problème que nous traiterons dans la section \ref{sec:Développement}.
\begin{deff}[Isomorphisme de graphe]
Soient $G, H$ deux graphes $G=(V, E)$ et $H=(V', E')$. Une fonction $f$ entre $ F $ et $ G $ est un isomorphisme ssi $ f $ est une bijection entre les sommets de $ G $ et de $ H $ t.q. $ (u,v) \in E$ ssi $ (f(u), f(v)) \in H$.
\end{deff}
Nous remarquons qu'il y a plusieurs propriétés qui doivent être satisfaites:
 \begin{enumerate}[label=(\roman*)]
 	\item $|V|=|V'|$,
 	\item $|E|=|E'|$,
 	\item $G, H$ ont le même nombre de composantes connectées.
 \end{enumerate}
Au début on peut penser que si tous les sommets des deux graphes ont le même dégrée, le graphe sont forcement isomorphes. 
La figure \ref{fig:nonisograph} montre un exemple de graphes dont tous les sommets ont le même degré mais qui ne sont pas isomorphes.
\begin{deff}[Isomorphisme de sous-graphe]
Soient $G, H$ deux graphes $G=(V, E)$ et $H=(V', E')$. Nous nous demandons s'il existe un sous-graphe $ G'=(V_s, E_s), V_s \subseteq V \text{ et } E_s \subseteq E \cap (V \times V) \text{ tq } G' =_{iso} H. $
\end{deff}
\begin{deff}[Plus grandes sous structures communes]
	
\end{deff}


\newpage
\section{Développement}
\label{sec:Développement}
\subsection{Isomorphisme de graphe}
\subsubsection{Modélisation}
Soient $G=(V, E),\ G'=(V', E')$ deux graphes. 
La définition d'isomorphisme nous permet d'obtenir directement une modélisation. Soit $CSP = (X, D, C)$ le problème de PPC que nous voulons définir. Nous définissons l'ensemble de variables de la façon suivante: 
$$X=\{x_v\ |\ v \in V \},\ \forall v \in V.$$
Puisque nous cherchons une fonction  $f: V \rightarrow V'$ t.q. $ (u,v) \in E$ ssi $ (f(u), f(v)) \in E'$, nous savons déjà que deux sommets $u \in V, u' \in V'$ peuvent être liés par $f$ uniquement si $deg(u) = deg(u')$. Donc, nous allons considérer pour chaque $u \in V$ l'ensemble des sommets $\{v_1, v_2..v_k\}$ tq $v_1, v_2..v_k \in V' \text{ et } deg(v_1)=deg(v_2)=..=deg(v_k)=deg(u), \text{ pour un certain } k: 0 \leq k \leq |V'|$. Nous remarquons tout de suite que si $k=0$ pour un sommet $u \in V$, il n'y aura pas une solution à notre problème. Pour chaque variable $x \in X,$ le domaine est ainsi définis comme l'ensemble: 
$$D_{x_v}=\{u\ |\ deg(u) = deg(v) \land u \in V'\}.$$
Pour définir les contraintes, nous suivrons la définition d'isomorphisme, donc $\forall u, v \in V \text{ tq } (u,v) \in E:$
$$C(x_u, x_v) \triangleq \text{ " pour chaque sommets } u' \in D_{x_u}: \exists v' \in D_{x_v} \text{ tq } (u', v') \in E' \land u' \neq v' \text{ "}.$$
\begin{rem}
	\label{rem:contraintes}
	Nous allons appeler ce type de contrainte \it contrainte d'adjacence des sommets\rm. Ici les contraintes sont binaires et définies localement. L'ensemble de contraintes locales ainsi définies peut être imaginé comme un seul contrainte globale $C_g$. À ce propos, il est utile de préciser que une contrainte globale \it alldiff \rm peut être utilisé une fois pour toute plutôt que indiquer $u' \neq v'$ pour chaque contrainte $C(x_u, x_v)$.
\end{rem}
\subsubsection{Résolution}
En considérant l'ensemble de contraintes locales comme une contrainte globale $C_g$, nous allons introduire maintenant le réseau des valeurs associé à $C_g$. Puisque $\forall x \in X: D_x \subseteq V'$ nous savons que
$$U(C)=\bigcup_{x \in X(C)} D_x = V'.$$
Nous savons aussi que $|X|=|V|,$ parce que une et une seule variable est associé à chaque sommet de $G$. Nous définissions le graphe bipartite $GV(C)=(X(C), U(C), A)$ où $A \subseteq X(C) \times U(C)$ est l'ensemble d'arêtes défini de la manière suivante:
$$A=\{(u, x_v) \in A\ |\ u \in U(C), x_v \in X(C) \land deg(v)=deg(u) \},$$
c'est à dire que dans le graphe $GV(C)$ il y aura un arc entre une variable $x_v$ et un sommet $u \in G'$ seulement si $v$ et $u$ ont le même degré. 
Pour construire le réseau de valeurs $N(C),$ nous devons introduire des nouveaux éléments:
\begin{enumerate}
	\item un sommet $s$ et un sommet $t$,
	\item un arc $(s, u)$ pour chaque $u \in U(C),$
	\item un arc $(x_v, t)$ pour chaque $x_v \in X(C),$
	\item une direction pour chaque arc $a=(u,v) \in A$: nous supposons que $a$ est sortant de $u$ et entrant dans $v$.
\end{enumerate}
 Le problème de trouver un isomorphisme peut être transformé en un problème de flot sur $N(C)$ en définissant la fonction de capacité $c: A \rightarrow [\mathbb{R}, \mathbb{R}],$ où  $[\mathbb{R}, \mathbb{R}]$ est un intervalle fermé dans lequel le flot qui passe dans l'arc $a$ peut prendre une valeur, borne inférieure et supérieure comprises. Dans notre cas, la fonction de capacité $c$ est définie par
\[  \forall (u,v) \in A:
c((u,v))= 
\begin{cases}
[1,1]& \text{if } u=s \lor v=t,\\
[0,1]              & \text{sinon}.
\end{cases}
\]
Lorsque nous injections un flot de $s$ à $u \in U(C)$ et puis de $u$ à $x_v \in X(C)$ nous allons couper les arcs qui sont incompatible avec la contrainte globale où $u=x_v$ est fixé: par exemple, si un sommet $v' \in V$ est adjacent à $v$, mais $u' \in V'$ n'est pas adjacent à $u$, l'arc $(x_{v'},u')$ sera coupé. Nous allons itérer cette procédure jusqu'à nous trouvions une solution. Si pendant le processus un sommet perd tous ses arcs (i.e. son degré devient $0$), nous recommençons la procédure de zéro mais en choisissant au début un arc différent de $(u, x_v).$
Un exemple de cette procédure de résolution pour le cas de figure \ref{fig:isographstar} est présenté dans la figure \ref{fig:algstar} . Nous insistons sur le fait que à chaque pas, des que un arc est choisi, il y a des arcs qui sont coupés à cause au moins d'une des contraintes suivantes:
\begin{enumerate}
	\item le contrainte \it alldiff, \rm
	\item le contrainte d'adjacence des sommets.
\end{enumerate}


\begin{figure*}[t!]
	\begin{subfigure}[t]{0.5\textwidth}
		\centering
		\includegraphics[height=1.6in]{alg0.eps}
		\caption{État initial}
	\end{subfigure}%
	~
	\begin{subfigure}[t]{0.5\textwidth}
		\centering
		\includegraphics[height=1.6in]{alg1.eps}
		\caption{$(x_a, a')$ choisi}
	\end{subfigure}
	
	\begin{subfigure}[t]{0.5\textwidth}
		\centering
		\includegraphics[height=1.6in]{alg2.eps}
		\caption{$(x_c, b')$ choisi}
	\end{subfigure}
	~
	\begin{subfigure}[t]{0.5\textwidth}
		\centering
		\includegraphics[height=1.6in]{alg3.eps}
		\caption{$(x_e, c')$ choisi}
	\end{subfigure}
	
	\begin{subfigure}[t]{0.5\textwidth}
		\centering
		\includegraphics[height=1.6in]{alg4.eps}
		\caption{$(x_b, d')$ choisi}
	\end{subfigure}
	~
	\begin{subfigure}[t]{0.5\textwidth}
		\centering
		\includegraphics[height=1.6in]{alg5.eps}
		\caption{$(x_d, e')$ choisi}
	\end{subfigure}
	\caption{\label{fig:algstar} Résolution pour l'exemple dans la figure \ref{fig:isographstar}.}
\end{figure*}




\newpage
\subsection{Isomorphisme de sous-graphe}
\subsubsection{Modélisation}
Comme pour le premier problème, nous pouvons définir un $CSP=(X, D, C)$ tout de suite en utilisant la définition du problème d'isomorphisme de sous-graphe que nous avons introduit dans la section 
\ref{sec:intro}. Soient $G=(V, E),\ G'=(V', E')$ deux graphes. Les variables et les domaines sont définies toujours de la manière suivante:
$$X=\{x_v\ |\ v \in V \},\ \forall v \in V,$$
$$D_{x_v}=\{u\ |\ deg(u) \leq deg(v) \land u \in V'\}.$$
Le degré de $v \in V$ doit être supérieur où égal au degré de $u \in V'$ parce que nous cherchons un sous-graphe de $G$ qui soit isomorphe à $G'$, donc si nous choisissons une couple $(x_v, u)$ où $deg(u) < deg(v),$ nous pouvons toujours supprimer $deg(v) - deg(u)$ arcs adjacent à $v$ pour obtenir un sous-graphe tq $deg(u) = deg(v).$
Comme avant, les contraintes peuvent être définies:
$$C(x_u, x_v) \triangleq \text{ " pour chaque sommets } u' \in D_{x_u}: \exists v' \in D_{x_v} \text{ tq } (u', v') \in E' \land u' \neq v' \text{ "}.$$
Les remarques \ref{rem:contraintes} s'appliquent toujours.
\subsubsection{Résolution}
Afin de déterminer une solution nous pouvons reprendre la construction du réseau de valeurs du premier problème en appliquant quelques petites modifications. Supposons que le réseau $N(C)=(X(C), U(C), A)$ est défini comme avant. Maintenant nous allons définir une nouvelle fonction de capacité des arcs:
\[  \forall (u,v) \in A:
c((u,v))= 
\begin{cases}
[1,1]& \text{if } u=s,\\
[0,1]              & \text{sinon}.
\end{cases}
\]
Donc maintenant les arcs entrants de $t$ ont capacité: $[0,1]$. Dans le sens de la modélisation du problème, ça signifie que le but n'est plus d'essayer de couvrir entièrement les sommets des deux graphes, mais seulement ceux du deuxième graphe, c'est à dire $H$. De cette façon, dès que nous réussissons à couvrir tous les sommets $u \in U(C)$ avec le flot, nous allons trouver une solution. 
\begin{figure}[t]
	\centering
	\includegraphics[height=1.2in]{sousgrapheisomorphisme.eps}
	\caption{\label{fig:isosubgraph}Exemple d'isomorphisme de sous-graphe}
\end{figure}
Comme avant, un exemple de résolution pour le cas présenté dans la figure \ref{fig:isosubgraph} est indiqué à la figure \ref{fig:sousalgstar} .



\begin{figure*}[t!]
	\begin{subfigure}[t]{0.5\textwidth}
		\centering
		\includegraphics[height=1.5in]{sousalg0.eps}
		\caption{État initial}
	\end{subfigure}%
	~
	\begin{subfigure}[t]{0.5\textwidth}
		\centering
		\includegraphics[height=1.5in]{sousalg1.eps}
		\caption{$(x_a, a')$ choisi}
	\end{subfigure}
	
	\begin{subfigure}[t]{0.5\textwidth}
		\centering
		\includegraphics[height=1.5in]{sousalg2.eps}
		\caption{$(x_b, b')$ choisi}
	\end{subfigure}
	~
	\begin{subfigure}[t]{0.5\textwidth}
		\centering
		\includegraphics[height=1.5in]{sousalg3.eps}
		\caption{$(x_d, c')$ choisi}
	\end{subfigure}
	
	\caption{\label{fig:sousalgstar} Résolution pour l'exemple dans la figure \ref{fig:isosubgraph}.}
\end{figure*}

\newpage
\subsubsection{Plus grandes sous structures communes}

\newpage
\section{Conclusion}


\newpage
\section{Bibliographie}
\end{document}
