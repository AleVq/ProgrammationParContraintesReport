\documentclass[french]{article}
\usepackage[T1]{fontenc}
\usepackage[utf8]{inputenc}
\usepackage{lmodern}
\usepackage[a4paper]{geometry}
\usepackage{babel}
\usepackage{enumitem}   
\usepackage{amssymb}
\usepackage{verbatim}
\usepackage{amsthm}
\usepackage{amsmath}
\usepackage{algorithm}
\usepackage{algorithmicx}
\usepackage{hyperref}
\usepackage{algpseudocode}
\usepackage{subcaption}
\usepackage[numbers]{natbib}
\usepackage{graphicx}
\theoremstyle{definition}
\newtheorem{deff}{D\'efinition}[section]
\newtheorem{rem}{Remarque}[section]
\graphicspath{{./images/}}
\theoremstyle{remark}
\newtheorem*{remark}{Remark}
\DeclareMathOperator*{\argmax}{arg\,max}
\DeclareMathOperator*{\argmin}{arg\,min}

\begin{document}
	\title{Programmation par Contraintes - Rapport}
	\author{Vasquez Alessandro}
	\maketitle
	\newpage
	\tableofcontents
	\newpage
\section{Introduction}
\label{sec:intro}
\begin{figure*}[t!]
	\begin{subfigure}[t]{0.5\textwidth}
		\centering
		\includegraphics[height=1.5in]{isomorphismesimple.eps}
		\caption{\label{fig:isographstar}Deux graphes isomorphes}
	\end{subfigure}%
	~ 
	\begin{subfigure}[t]{0.5\textwidth}
		\centering
		\includegraphics[height=1.8in]{nonisomorphegraphes.eps}
		\caption{\label{fig:nonisograph}Graphes  non-isomorphes}
	\end{subfigure}
	\caption{Isomorphisme, examples}
\end{figure*}

Nous verrons d'abord les définitions préliminaires dans la section \ref{sec:intro}. Ensuite nous allons traiter les problème d'isomorphisme de graphe et sous-graphe dans la section \ref{sec:Développement}. 
Enfin dans la section \ref{sec:conc} nous citerons des modèles différents proposés par différents auteur. 

\begin{deff}[Graphe]
	Un graphe $G$ est un couple $(V,E)$ où $V$ est l'ensemble de sommets et $E$ est l'ensemble d'arêtes. Une arête est une paire $(u,v)$ de sommets $u,v \in V \text{ tq } u \neq v.$
	
\end{deff}

\begin{deff}[Isomorphisme de graphe]
Soient $G, H$ deux graphes $G=(V, E)$ et $H=(V', E')$. Une fonction $f$ entre $ F $ et $ G $ est un isomorphisme ssi $ f $ est une bijection entre les sommets de $ G $ et de $ H $ t.q. $ (u,v) \in E$ ssi $ (f(u), f(v)) \in H$.
\end{deff}
Nous remarquons qu'il y a plusieurs propriétés qui doivent être satisfaites au même temps:
 \begin{enumerate}[label=(\roman*)]
 	\item $|V|=|V'|$,
 	\item $|E|=|E'|$,
 	\item $G, H$ ont le même nombre de composantes connectées.
 \end{enumerate}
Au début on peut penser que si tous les sommets des deux graphes ont le même dégrée, le graphe sont forcement isomorphes. 
La figure \ref{fig:nonisograph} montre un exemple de graphes dont tous les sommets ont le même degré mais qui ne sont pas isomorphes.
\begin{deff}[Isomorphisme de sous-graphe]
Soient $G, H$ deux graphes $G=(V, E)$ et $H=(V', E')$. Nous nous demandons s'il existe un sous-graphe $ G'=(V_s, E_s), V_s \subseteq V \text{ et } E_s \subseteq E \cap (V \times V) \text{ tq } G' =_{iso} H. $
\end{deff}
\begin{deff}[Plus grandes sous structures communes]
Entre les sous structures communes d'un graphe il y a différents types de sous-graphes sur lesquels nous pouvons tomber. Soit $G=(V,E)$ un graphe et $G'=(V',E')$ un sous-graphe de $G$. Alors:
\begin{enumerate}
	\item $G'$ est sous-graphe couvrant (ou \it graphe partiel \rm) de $G$ si $V=V',$
	\item $G'$ est \it sous-graphe partiel \rm de $G$ si $V' \subseteq V \land \forall v,u \in V' : (u,v) \in E'  \text{ ssi } (u,v) \in E.$ 
\end{enumerate}
La sous-structure commune de deux graphe $G, H$ peut être définie comme un graphe $g$ tq il existe un isomorphisme de sous-graphe de $G$ à $g$ et de $H$ à $g$.
\end{deff}
\begin{rem}
	Nous supposons que les isomorphisme sont définies sur graphes non orientés.
\end{rem}

\newpage
\section{Développement}
\label{sec:Développement}
\subsection{Isomorphisme de graphe}
\subsubsection{Modélisation}
\label{sssec:model1}
Soient $G=(V, E),\ G'=(V', E')$ deux graphes. 
La définition d'isomorphisme nous permet d'obtenir directement une modélisation. Soit $CSP = (X, D, C)$ le problème de PPC que nous voulons définir. Nous définissons l'ensemble de variables de la façon suivante: 
$$X=\{x_v\ |\ v \in V \},\ \forall v \in V.$$
Puisque nous cherchons une fonction  $f: V \rightarrow V'$ t.q. $ (u,v) \in E$ ssi $ (f(u), f(v)) \in E'$, nous savons déjà que deux sommets $u \in V, u' \in V'$ peuvent être liés par $f$ uniquement si $deg(u) = deg(u')$. Donc, nous allons considérer pour chaque $u \in V$ l'ensemble des sommets $\{v_1, v_2..v_k\}$ tq $v_1, v_2..v_k \in V' \text{ et } deg(v_1)=deg(v_2)=..=deg(v_k)=deg(u), \text{ pour un certain } k: 0 \leq k \leq |V'|$. Nous remarquons tout de suite que si $k=0$ pour un sommet $u \in V$, il n'y aura pas une solution à notre problème. Pour chaque variable $x \in X,$ le domaine est ainsi définis comme l'ensemble: 
$$D_{x_v}=\{u\ |\ deg(u) = deg(v) \land u \in V'\}.$$
Pour définir les contraintes, nous suivrons la définition d'isomorphisme, donc $\forall u, v \in V \text{ tq } (u,v) \in E:$
$$c(x_u, x_v) \triangleq \text{ " pour chaque sommets } u' \in D_{x_u}: \exists v' \in D_{x_v} \text{ tq } (u', v') \in E' \land u' \neq v' \text{ "}.$$
\begin{rem}
	\label{rem:contraintes}
	Nous allons appeler ce type de contrainte \it contrainte d'adjacence des sommets\rm. Ici les contraintes sont binaires et définies localement. L'ensemble de contraintes locales ainsi définies peut être imaginé comme un seul contrainte globale $C$. À ce propos, il est utile de préciser que une contrainte globale \it alldiff \rm peut être utilisé une fois pour toute plutôt que indiquer $u' \neq v'$ pour chaque contrainte $c(x_u, x_v)$.
\end{rem}
\subsubsection{Complexité du modèle}
\label{subsubsec:compl1}
La complexité du problème $CSP=(X,D,C)$ est donné par le nombre de variables, contraintes et par la taille des domaines. Pour ce problème nous avons que par construction $|X|=|V|$, c'est à dire que le nombre de variables est égal au nombre de sommets de $G$. Chaque domaine est défini comme un sous-ensemble de l'ensemble de sommets de $G'$, donc $|D_x| \leq |V'|, \forall x \in X.$ Puisque les contraintes sont définies comme l'ensemble $C \subseteq (V \times V'),$ le nombre de contraintes est toujours $|C| \leq |V| \cdot |V'|.$

\subsubsection{Résolution}
En considérant l'ensemble de contraintes locales comme une contrainte globale $C$, nous allons introduire maintenant le réseau des valeurs \cite{regin} associé à $C$. Puisque $\forall x \in X: D_x \subseteq V'$ nous savons que
$$U(C)=\bigcup_{x \in X(C)} D_x = V'.$$
Nous savons aussi que $|X|=|V|,$ parce que une et une seule variable est associé à chaque sommet de $G$. Nous définissions le graphe bipartite $GV(C)=(X(C), U(C), A)$ où $A \subseteq X(C) \times U(C)$ est l'ensemble d'arêtes défini de la manière suivante:
$$A=\{(u, x_v) \in A\ |\ u \in U(C), x_v \in X(C) \land deg(v)=deg(u) \},$$
c'est à dire que dans le graphe $GV(C)$ il y aura un arc entre une variable $x_v$ et un sommet $u \in G'$ seulement si $v$ et $u$ ont le même degré. 
Pour construire le réseau de valeurs $N(C),$ nous devons introduire des nouveaux éléments:
\begin{enumerate}
	\item un sommet $s$ et un sommet $t$,
	\item un arc $(s, u)$ pour chaque $u \in U(C),$
	\item un arc $(x_v, t)$ pour chaque $x_v \in X(C),$
	\item une direction pour chaque arc $a=(u,v) \in A$: nous supposons que $a$ est sortant de $u$ et entrant dans $v$.
\end{enumerate}
 Le problème de trouver un isomorphisme peut être transformé en un problème de flot sur $N(C)$ en définissant la fonction de capacité $c: A \rightarrow [\mathbb{R}, \mathbb{R}],$ où  $[\mathbb{R}, \mathbb{R}]$ est un intervalle fermé dans lequel le flot qui passe dans l'arc $a$ peut prendre une valeur, borne inférieure et supérieure comprises. Dans notre cas, la fonction de capacité $c$ est définie par
\[  \forall (u,v) \in A:
c((u,v))= 
\begin{cases}
[1,1]& \text{if } u=s \lor v=t,\\
[0,1]              & \text{sinon}.
\end{cases}
\]
L'idée c'est de trouver un flot tq l'arc $(t,q)$ est saturé. L'algorithme pour le trouver est décrit in \cite{regin}.
Lorsque nous injections un flot de $s$ à $u \in U(C)$ et puis de $u$ à $x_v \in X(C)$ nous allons couper les arcs qui sont incompatible avec la contrainte globale où $u=x_v$ est fixé: par exemple, si un sommet $v' \in V$ est adjacent à $v$, mais $u' \in V'$ n'est pas adjacent à $u$, l'arc $(x_{v'},u')$ sera coupé. Ici un algorithme d'arc-consistance peut être utilisé. Ensuite nous allons itérer cette procédure jusqu'à nous trouvions une solution. Si pendant le processus un sommet perd tous ses arcs (i.e. son degré devient $0$), nous recommençons la procédure de zéro mais en choisissant au début un arc différent de $(u, x_v).$
Un exemple de cette procédure de résolution pour le cas de figure \ref{fig:isographstar} est présenté dans la figure \ref{fig:algstar} . Nous insistons sur le fait que à chaque pas, dès que un arc est choisi, il y a des arcs qui sont coupés à cause au moins d'une des contraintes suivantes:
\begin{enumerate}
	\item les contraintes \it alldiff, \rm
	\item les contraintes d'adjacence des sommets.
\end{enumerate}
Par exemple, dès que un flot est injecté sur le chemin $s \rightarrow a' \rightarrow x_a \rightarrow t \rightarrow s$, nous coupons tous les autres arcs qui partent de $a'$ et ceux qui arrivent en $x_a$ pour la contrainte \it alldiff. \rm Ensuite pour les contraintes d'adjacence des sommets nous considérons que
\begin{enumerate}
	\item $neighbors(a)=\{c, d\},$
	\item $neighbors(a')=\{b', e'\}.$
\end{enumerate}
Donc nous allons couper les arcs

$$ a=(b', x_v) \text{ tq } v \notin \{c,d\} , $$
$$ a=(e', x_v) \text{ tq } v \notin \{c,d\}, $$
$$ a=(u, c) \text{ tq } u \notin \{b',e'\}, $$
$$ a=(u, d) \text{ tq } u \notin \{b',e'\}. $$
Ensuite nous allons itérer la procédure sur un autre sommets jusqu'à le flot qui entre dans $s$ est égal à $|U(C)|,$ i.e. tous les sommet de $G'$ sont couverts. 

\subsubsection{Résolution, complexité}
Sans compter l'initialisation des structures de données, l'algorithme de résolution va essayer d'injecter le flot pour toutes les permutations admissibles des couples $(u, x_v), u \in U(C), x \in X(C)$, dans le pire des cas. Chaque fois que le flot est injecté, nous allons chercher des arcs à couper en utilisant par exemple un algorithme d'arc-consistance. Supposons que $AC$ est la complexité de l'algorithme d'arc-consistance.  
Alors la complexité sera $$O(AC \cdot (|X(C)|^2 \cdot |U(C)|)).$$
En fonction de quel algorithme d'arc-consistance et quel structure de données ont été choisis, la complexité de l'algorithme peut arriver à être exponentielle, au pire.
Par contre, nous savons que l'algorithme de résolution va s'arrêter dés qu'une solution a été trouvée. Il y a beaucoup d'exemples (comme dans la figure \ref{fig:algstar}) dans lesquels l'algorithme va trouver une solution en temps polynomial par rapport à l'ensemble des sommets.
\begin{comment}
\begin{algorithm}
	\caption{Résolution du réseau de valeurs}
	\begin{algorithmic}[1]
		\State $Q \gets U(C)$
		\State $Q' \gets X(C)$
		\While { $Q$ not empty }
		\State $u \gets dequeue(Q)$
		\State $x_v \gets dequeue(Q')$
		\State $Sol \gets (u, x_v) $ 
		\ForAll{$a \in A$}
		\If { $a$ doesn't respect \it alldiff \rm or adjacency constraint}
		\State delete $a$
		\EndIf
		\EndFor
		\EndWhile
	\end{algorithmic}
\end{algorithm}
\end{comment}

%\cite{DBLP:books/daglib/0023376}

\begin{figure*}[t!]
	\begin{subfigure}[t]{0.5\textwidth}
		\centering
		\includegraphics[height=1.6in]{alg0.eps}
		\caption{État initial}
	\end{subfigure}%
	~
	\begin{subfigure}[t]{0.5\textwidth}
		\centering
		\includegraphics[height=1.6in]{alg1.eps}
		\caption{$(a', x_a)$ choisi}
	\end{subfigure}
	
	\begin{subfigure}[t]{0.5\textwidth}
		\centering
		\includegraphics[height=1.6in]{alg2.eps}
		\caption{$(b', x_c)$ choisi}
	\end{subfigure}
	~
	\begin{subfigure}[t]{0.5\textwidth}
		\centering
		\includegraphics[height=1.6in]{alg3.eps}
		\caption{$(c', x_e)$ choisi}
	\end{subfigure}
	
	\begin{subfigure}[t]{0.5\textwidth}
		\centering
		\includegraphics[height=1.6in]{alg4.eps}
		\caption{$(d', x_b)$ choisi}
	\end{subfigure}
	~
	\begin{subfigure}[t]{0.5\textwidth}
		\centering
		\includegraphics[height=1.6in]{alg5.eps}
		\caption{$(e', x_d)$ choisi}
	\end{subfigure}
	\caption{\label{fig:algstar} Résolution pour l'exemple dans la figure \ref{fig:isographstar}.}
\end{figure*}




\newpage
\subsection{Isomorphisme de sous-graphe}
\subsubsection{Modélisation}
Comme pour le premier problème, nous pouvons définir un $CSP=(X, D, C)$ tout de suite en utilisant la définition du problème d'isomorphisme de sous-graphe que nous avons introduit dans la section 
\ref{sec:intro}. Soient $G=(V, E),\ G'=(V', E')$ deux graphes. Les variables et les domaines sont définies toujours de la manière suivante:
$$X=\{x_v\ |\ v \in V \},\ \forall v \in V,$$
$$D_{x_v}=\{u\ |\ deg(u) \leq deg(v) \land u \in V'\}.$$
Le degré de $v \in V$ doit être supérieur où égal au degré de $u \in V'$ parce que nous cherchons un sous-graphe de $G$ qui soit isomorphe à $G'$, donc si nous choisissons une couple $(x_v, u)$ où $deg(u) < deg(v),$ nous pouvons toujours supprimer $deg(v) - deg(u)$ arcs adjacent à $v$ pour obtenir un sous-graphe tq $deg(u) = deg(v).$
Comme avant, les contraintes peuvent être définies:
$$C(x_u, x_v) \triangleq \text{ " pour chaque sommets } u' \in D_{x_u}: \exists v' \in D_{x_v} \text{ tq } (u', v') \in E' \land u' \neq v' \text{ "}.$$
Les remarques \ref{rem:contraintes} s'appliquent toujours.

\subsubsection{Complexité du modèle}
Puisque nous avons défini la base du problème en utilisant la modélisation pour l'isomorphisme de graphe, les remarques sur la complexité dans le paragraphe \ref{subsubsec:compl1} s'appliquent toujours.

\subsubsection{Résolution}
\label{sssec:res}
Afin de déterminer une solution nous pouvons reprendre la construction du réseau de valeurs du premier problème en appliquant quelques petites modifications. Supposons que le réseau $N(C)=(X(C), U(C), A)$ est défini comme avant. Maintenant nous allons définir une nouvelle fonction de capacité des arcs:
\[  \forall (u,v) \in A:
c((u,v))= 
\begin{cases}
[1,1]& \text{if } u=s,\\
[0,1]              & \text{sinon}.
\end{cases}
\]
Donc maintenant les arcs entrants de $t$ ont capacité: $[0,1]$. Dans le sens de la modélisation du problème, ça signifie que le but n'est plus d'essayer de couvrir entièrement les sommets des deux graphes, mais seulement ceux du deuxième graphe, c'est à dire $G'$. De cette façon, dès que nous réussissons à couvrir tous les sommets $u \in U(C)$ avec le flot, nous allons trouver une solution. 
\begin{figure}[t]
	\centering
	\includegraphics[height=1.2in]{sousgrapheisomorphisme.eps}
	\caption{\label{fig:isosubgraph}Exemple d'isomorphisme de sous-graphe}
\end{figure}
Comme avant, un exemple de résolution pour le cas présenté dans la figure \ref{fig:isosubgraph} est indiqué à la figure \ref{fig:sousalgstar} .

\subsubsection{Résolution, complexité}
Comme avant, la complexité va toujours être 
 $$O(AC \cdot (|X(C)|^2 \cdot |U(C)|)),$$
avec une complexité exponentielle, toujours en fonction de quel algorithme d'arc-consistance et quel structure de données ont été choisis. Par contre, dans ce cas, puisque nous allons nous arrêter dès que nous trouvons un sous-graphe de $G$ pour lequel il y a une solution, l'algorithme de résolution peut être plus vite de l'algorithme pour l'isomorphisme simple. De toute façon l'idée c'est que un grand nombre d'exemples peut être trouver pour montrer que ce algorithme arrive à trouver une solution en temps polynomial.


\begin{figure*}[t!]
	\begin{subfigure}[t]{0.5\textwidth}
		\centering
		\includegraphics[height=1.5in]{sousalg0.eps}
		\caption{État initial}
	\end{subfigure}%
	~
	\begin{subfigure}[t]{0.5\textwidth}
		\centering
		\includegraphics[height=1.5in]{sousalg1.eps}
		\caption{$(a', x_a)$ choisi}
	\end{subfigure}
	
	\begin{subfigure}[t]{0.5\textwidth}
		\centering
		\includegraphics[height=1.5in]{sousalg2.eps}
		\caption{$(b', x_b)$ choisi}
	\end{subfigure}
	~
	\begin{subfigure}[t]{0.5\textwidth}
		\centering
		\includegraphics[height=1.5in]{sousalg3.eps}
		\caption{$(c', x_d)$ choisi}
	\end{subfigure}
	
	\caption{\label{fig:sousalgstar} Résolution pour l'exemple dans la figure \ref{fig:isosubgraph}.}
\end{figure*}

\newpage
\subsection{Plus grandes sous structures communes}
\subsubsection{Introduction}
Grâce à la modélisation utilisée pour les problèmes de isomorphisme de graphe et sous-graphe, nous allons tout de suite arriver à une solution. Nous assumons que $CSP=(X,D,C)$ est le problème de programmation par contrainte défini comme dans \ref{sssec:model1} et que $N(C)$ est le réseau de valeurs défini comme dans \ref{sssec:res}.

\subsubsection{Résolution}
Une solution à ce problème peut être trouvée en itérant l'algorithme de résolution du réseau de valeurs et en essayant tous les cas et en gardant toutes les solutions admissibles dans un ensemble $S=\{S_0, S_1..\},$ où chaque $S_i$ est l'ensemble de couple $\{ (u_0, x_{v_0}), (u_1, x_{v_1}).. \}$. Enfin nous allons simplement choisir la solution qui correspond à l'isomorphisme de sous-graphe dont l'ensemble de sommets est le plus grand, c'est à dire:
\[
\argmax_{|S_i|} (S).
\]

\subsubsection{Résolution, complexité}
Avec cette tactique, l'algorithme de résolution va explorer toujours tous les possibles solutions, c'est sûr que la complexité sera exponentielle par rapport au nombre de sommets, peu importe quel algorithme d'arc-consistance ou structure de données sont utilisés. 

\newpage
\section{Conclusion}
\label{sec:conc}
Nous avons présenté dans la section \ref{sec:Développement} trois simples modélisations et un algorithme de résolution qui utilise le réseau de valeurs pour trouver une solution. Malheureusement, la complexité de cet algorithme est exponentielle (dans le pire des cas) et, pour le problème du plus grande sous structure commune, elle est même exponentielle pour tout le cas. Tout à fait, ces problèmes sont encore ouverts, même s'il y a déjà nombreux articles qui proposent des nouveaux algorithmes et modélisations afin de réduire la complexité, par exemple \cite{SorlinS082010,SorlinS08, TakapouiB16}. Le problème c'est que il y a déjà beaucoup d'algorithmes qui arrivent à avoir une complexité quasi-polynomial pour un certain nombre d'exemples, mais il sont toujours exponentiels dans le pire des cas. Donc, le but maintenant c'est trouver un algorithme ou une modélisation tells que la complexité soit réduite pour tous les cas, c'est à dire garantir une complexité plus basse même dans le pire des cas.



\newpage
\bibliographystyle{plain}
\bibliography{mybib}
\end{document}
